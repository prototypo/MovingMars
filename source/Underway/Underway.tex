%%%%%%%%%%%%%%%%%%%%%%%%%%%%%%%%%%%%%%%%%%%%%%%%%%
%
% Chapter:  Underway
%
%%%%%%%%%%%%%%%%%%%%%%%%%%%%%%%%%%%%%%%%%%%%%%%%%%

\chapter{Underway}

The spacecraft awoke.

Albatenius received a new command, and unthinkingly executed it. The batteries had stopped taking a charge from the solar panels. No more electrical potential could be acquired. It would have to be enough.

The two wings of solar panels twisted to make their angle almost parallel to the Sun's rays. Two slight nudges from the remaining gyroscope twisted the spacecraft farther around. The thin solar wind would impact the spacecraft just a bit less. It, too, would have to be enough.

No spacecraft could sit forever in an L1 point, where the Sun's and Earth's gravity were felt in equal amounts. Staying exactly in L1 was like sitting on the top of a long pole. The slightest wavering in one direction or another would cause the gravitational tug of either the Sun or the Earth to dominate. So spacecraft orbited the point in a so-called halo orbit, circling around L1 as if it were itself a gravity well.

Albatenius' solar panels moved again, back to face the Sun. A quarter of an orbit later, they moved perpendicular again. And so it went. Back and forth, each movement changing the tiny force of the solar wind. Days passed, then weeks. The tiny halo orbit began to weaken, to elongate. Finally, it broke. Albatenius sailed free of the L1 point, and slowly into a highly elliptical orbit around Earth.

Another command realigned the solar panels to charge the batteries back to their full potential. Another waiting game ensued. Coaxing a nearly dead spacecraft into travel was a subtle game, full of careful consideration, careful timing, and even more careful use of scarce resources.


% If the chapter ends in an odd page, you may want to skip having the page
%  number in the empty page
\newpage
\thispagestyle{empty}
