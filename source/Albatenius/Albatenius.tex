%%%%%%%%%%%%%%%%%%%%%%%%%%%%%%%%%%%%%%%%%%%%%%%%%%
%
% Chapter:  Albatenius
%
%%%%%%%%%%%%%%%%%%%%%%%%%%%%%%%%%%%%%%%%%%%%%%%%%%

\chapter{Albatenius}

The spacecraft slept.

Its Italian builders had given it a short, Latinised name. That was probably called for, since the ninth century solar astronomer for which it was named, Ab\=u 'Abd All\={a}h Mu\d{h}ammad ibn J\={a}bir ibn Sin\={a}n al-Raqq\=\i{} al-\d{H}arr\={a}n\=\i{} a\d{s}-\d{S}\={a}bi' al-Batt\={a}n\=\i{}, did have a very long name. They called it, echoing their medieval ancestors, Albatenius.

Albatenius was well past its mission observing the Sun. Its solar arrays had been heavily bombarded by the stellar wind, the charged particles had damaged the silicon junctions that produced electrical power. Its batteries, after years of hysteresis, could no longer retain much electrical potential. The few sips of hydrazine remaining in its thruster tanks were frozen solid against the aluminium walls. Albatenius was little more than space junk, listing sideways, spinning slowly in the vacuum of space.

The spacecraft awoke.

Waking might be too strong a metaphor. A signal was received, processed, and responded to. No one had called in many years. But the radiation-hardened computer still operated, as did one of the radio transceivers. The batteries held enough power for some limited operation.

Current passed through copper traces etched onto its circuit boards. No motors were actuated, no motion was caused. But there was action. Inventory was taken. Status was checked. Slowly, carefully, always aware of the exact amount voltage drained from the aged batteries, Albatenius was stirred by an unseen hand.

No attempt was made to investigate the scientific instruments. They were hardly important now. The spacecraft had measured the ion and electron composition of the solar wind for an entire 7-year solar cycle. Other spacecraft had done the same. It was enough to guess what the flux would be at the first Lagrange point of the Earth-Sun system. L1 was a well-studied region of space.

The x-axis gyroscope was known to be frozen. The z-axis one, too. The operator didn't dare waste the few precious volts remaining in the batteries. There was no attempt to restore them. The remaining two gyroscopes were touched ever so gently in turn. A hint of voltage was applied to the y-axis gyroscope. Its sensors reported no movement. A pause. Finally, as if knowing the finality of the operation, another hint of voltage was applied to the off-axis gyroscope. Nothing. Another pause.

With nothing to lose, a stronger voltage tickled the off-axis gyroscope. A wiggle, so slight it could have almost been sensor noise, registered. There was a bit of life in the old girl yet. Another pause.

Full voltage was applied to the last gyroscope for a few seconds. The gyroscope rotated just enough to start the spacecraft spinning slowly. It took every last bit of the remaining battery power to spin the last gyroscope a few more precious seconds for an equal but opposite spin. Albatenius slowed to a halt, the same as it was with a tiny but important difference. Its damaged solar arrays now pointed more or less toward the Sun. The batteries began, slowly but inexorably, to charge.


% If the chapter ends in an odd page, you may want to skip having the page
%  number in the empty page
\newpage
\thispagestyle{empty}
